% Final year project - Interim report
% Author: Alexander Jeffery

\documentclass{article}
\title{\textbf{Interim Report}: An Interpreter/Compiler for a Functional Programming Language}
\author{Alexander Jeffery}

\begin{document}
\maketitle

\section{Introduction}

Conventional programming languages are designed to solve problems by decomposing them into a list of instructions, which are executed by a computer in a specified sequence. They are said to be \emph{imperative}. This imperative principle of sequential execution is ultimately derived from the nature of computer architecture, which is imperative in nature. Such languages have become increasingly sophisticated over many years, abstracting the computer's hardware, so that programmers need not concern themselves with understanding the intricate details of the computer, which are often irrelevant to the problem they are solving. Imperative languages do not, however, abstract sequential execution (by definition).
\\
Functional programming languages differ in that they achieve computation by implementing the lambda calculus.

\section{Background}

\section{Methodology}

\section{Project Plan}

\end{document}
